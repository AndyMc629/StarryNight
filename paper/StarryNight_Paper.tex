%JACS TEMPLATE STARTS HERE
\documentclass[journal=jacsat,manuscript=communication]{achemso}
\usepackage[version=3]{mhchem} % Formula subscripts using \ce{}
\usepackage{lineno,xcolor}
\newcommand*{\mycommand}[1]{\texttt{\emph{#1}}}

%\documentclass[aip,graphicx]{revtex4-1}
%\documentclass[aip,apl,amsmath,amssymb,linenumbers,reprint]{revtex4-1}
%\documentclass[aip,apl,amsmath,amssymb,linenumbers,preprint]{revtex4-1}
\usepackage{graphicx}
\usepackage[version=3]{mhchem} % Formula subscripts using \ce{}
\usepackage{subfig}
\usepackage{dcolumn}% Align table columns on decimal point

\usepackage{amsmath}% amsmath...
\usepackage{bm}% bold math

\usepackage{siunitx}% JMF addition - once a physicist, always a physicist

%\bibliographystyle{aipnum4-1}
%\draft % marks overfull lines with a black rule on the right

\title{Starry Night: Or how I learned to stop worrying and love perovskites}

\author{Jarvist M. Frost}
\author{Keith T. Butler}
%\author{Federico Brivio}
%\author{Christopher H. Hendon}
\affiliation{Centre for Sustainable Chemical Technologies and Department of Chemistry, University of Bath, Claverton Down, Bath BA2 7AY, UK}

%\author{Mark van Schilfgaarde}
%\affiliation{Department of Physics, Kings College London, London WC2R 2LS, UK}

\author{Aron Walsh}
\email{a.walsh@bath.ac.uk}
\affiliation{Centre for Sustainable Chemical Technologies and Department of Chemistry, University of Bath, Claverton Down, Bath BA2 7AY, UK}

\begin{document}

\begin{abstract}
Abstract. EPSRC gave us some money so we did our best to do great science, and these are our conclusions. 
\end{abstract}

%\pacs{88.40.-j, 71.20.Nr, 72.40.+w, 61.66.Fn}
% 71.20.Nr 	Semiconductor compounds 
% 72.40.+w 	Photoconduction and photovoltaic effects
% 61.66.Fn 	Inorganic compounds 
% 88.40.jn 	Thin film Cu-based I-III-VI2 solar cells
% 88.40.-j 	Solar energy

%\maketitle 

\textbf{Introduction}

Solar cells based on from hybrid perovskites  display unusual device physics.
One unusual aspect is the notably hysterisis in current voltage curves, depending on rate of measurement and starting point within the curve. 
In addition, the highest apparent efficiencies are generated after having soaked the solar cell in forward bias at $1.4$ \si{V}.

Perovskite structures have a large potential dielectric constant due to the relative ease of distorting the cell structure resulting in an overall dipole between the A and B perovskite sites. 
Inorganic perovskites are a well known class of ferroelectrics, the  [TODO: Check].
The hybrid perovskite material, as compared to inorganic perovskites, has the additional possibility of the dipolar cation reorientating giving rise to induced domains.

In this paper we will
\begin{itemize}
 \item suggest how the unique behaviour of hybrid perovskite solar cells arrises from ferroelectric domains
 \item develop a monte-carlo code to model the ferroelectric behaviour of hybrid perovskites from a model classical Hamiltonian
 \item use this code to simulate device physics, hysterisis and built-in potential as a function of temperature
\end{itemize}

Work by Poglitsch and Weber measured the complex dielectric response of
methyl-ammonium lead halides (iodide, chloride, bromide) as a function of
temperature between $100-300$ \si{\kelvin}. 
\cite{poglitsch_dynamic_1987}
They observe a picosecond scale relaxation of the dielectric constant. 
They model these data with a analytic model based on considering the eight
possible diagonal alignments of the dipolar molecule within the cubic unit
cell. 
This model is then used to construct a partition function for the system with
three next-neighbour contributions, which is solved to indicate that a single
dipole-active relaxation occurs and thus a well defined Debye-like single
relaxation time exists in this complex system.  

\textbf{Method}

We start from the lattice dynamical theory of ferroelectricity, from the ideas of P.W. Anderson\cite{anderson_career_1994} and W.Cochran\cite{cochran_crystal_1960,cochran_crystal_1961}. 

We limit ourself to considering that the one unstable phonon mode in the system is the free rotation of the molecular dipole within a cubic perovskite structure. 
This free dipole is very similar to the 'rattling titanium' hypothesis, 

The treatment of polarisation as an effect of rotational brownian motion is analytically difficult\cite{mcconnell_rotational_1980}.
Here we simulate this physics directly with a monte-carlo method to calculate the equilibrium configuration of the dipoles.

We construct a model Hamiltonian from local (cage strain), applied electrostatic field and interacting dipole energies.

\begin{align}
H = &\sum^n_{dipole,E-field} &\frac{1}{4.\pi \epsilon_0} (p_i.E_0) \\
+ &\sum^{n,m}_{dipole,dipole} &\frac{1}{4.\pi \epsilon_0} (\frac{p_i.p_j}{r^3}-\frac{3(\hat{n}.p_i)(\hat{n}.p_j)}{r^3}) \\
+ &\sum^n_{dipole,strain} &K.|p_i.\hat{x}|
\end{align}  

Our Monte-Carlo method progresses with a Metropolis algorithm. 
A random lattice position is chosen, and a random new direction for the cation molecular dipole. 
The energy change $\Delta E$ is computed in a numerically efficient manner with a cut-off for the dipole-dipole interaction of two lattice units; the interaction energy is calculated for unshielded ($\epsilon_r =1$) dipole-dipole interactions. 
Exothermic moves are autuomatically accepted; endothermic moves are accepted if $\gamma < e^{-\beta \Delta E}$ where $\gamma$ is a random variable on $[0,1]$.

At an equilibrium, we associate an electric displacement $D$ related to $E_0$ and the polarisation density by:

\begin{equation}
D =  E_0 + 4\pi P
\end{equation}

The polarisation density $P$ we can calculate by a summation over the microscopic dipoles. 

\begin{equation}
\epsilon_s = 1+ \frac{4\pi P}{E_0}
\end{equation}

Here $\epsilon_s$ is a static relative permitivity, rather than dielectric constant, as it is a function of temperature.

\textbf{Results}

\textbf{Conclusion}

\textbf{Computational Details}

We used a Mersenne Twister\cite{matsumoto_mersenne_1998} pseudo random number generator. The initial state was a lattice of randomised dipoles.

%\begin{figure*}[ht!]
%\begin{center}
%\resizebox{15 cm}{!}{\includegraphics*{dipoles.pdf}}
%\caption{\label{fig-dip} Schematic perovskite crystal structure of MAPbI$_3$ (\textbf{a}), and the possible orientations of molecular dipoles within the lattice (\textbf{b}). Note that MA has an associated molecular dipole of 2.3 Debye, a fundamental difference compared to the spherical cation symmetry in inorganic perovskites such as CsSnI$_3$.} 
%\end{center}
%\end{figure*}

\begin{acknowledgement}
We acknowledge membership of the UK's HPC Materials Chemistry Consortium, which is funded by EPSRC grant EP/F067496. 
J.M.F. and K.T.B. are funded by EPSRC Grants EP/K016288/1 and EP/J017361/1, respectively.
%F.B. is funded through the EU DESTINY Network (Grant 316494).
%C.H.H. is funded by ERC (Grant 277757). 
A.W. acknowledges support from the Royal Society and ERC (Grant 277757). 
We are grateful for the lyrical encouragment of Salt N Pepa. 
\end{acknowledgement}

\begin{suppinfo}
    The codes, \textsc{StarryNight}, are available as a source code repository on GitHub\cite{GitHub}.
\end{suppinfo}

\bibliography{library}

\end{document}
